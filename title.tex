% file: title.tex

\begin{titlepage}
\begin{center}
  \hfill\\
  \vspace{1cm}
  % title of this document
  {\fontsize{36pt}{40pt}\NotoSansSCBold{} 神经网络与深度学习}\\
  \vspace{1em}
  {\LARGE\RobotoRegular{} \href{http://neuralnetworksanddeeplearning.com/index.html}{Neural Networks and Deep Learning}}\\
  \vspace{1cm}

  % borrow the code from http://latex-community.org/know-how/513-tikz-math
  \begin{tikzpicture}[scale=.06pt]
    \tikzmath{
      % --------------------------
      % the parameters of the tree
      % --------------------------
      \power=2.5; % the scale base factor
      \deviation=44.5; % the angle between the 3 child edges
      \numsteps=5; % number of levels
      let \startcolor=magenta; % the start color
      let \endcolor=black; % the end color
      % -------------------------------------------------------------------------
      % the function that draw one edge and call itself to draw the 3 child edges
      % -------------------------------------------------------------------------
      function Branch(\x,\y,\rotate,\step){
        \step=\step-1; % stops drawing if step < 0
        if (\step >= 0) then {
          \mix = int(100*\step/(\numsteps-1)); % the color mix parameter is in [0,100]
          \scale = \power^\step; % the scale factor
          { % "print" the tikz command that draw the edge
            \draw[shift={(\x pt,\y pt)}, rotate=\rotate, scale=\scale,
                  color=\startcolor!\mix!\endcolor, line width=\scale*.1 pt,
                  line cap=round]
              (0,0)--(1,0) coordinate(newbase);
          };
          coordinate \b; \b1 = (newbase); % the new base point
          for \a in {-\deviation,0,\deviation}{
            Branch(\bx1,\by1,mod(\rotate+\a,360),\step); % draw one child edge
          };
        };
      };
      % ----------------------
      % draw the four branches
      % ----------------------
      for \angle in {0,45,...,360}{
        Branch(0,0,\angle,\numsteps);
      };
    }
  \end{tikzpicture}
  
  \vspace{1cm}
  {\LARGE (美)\href{http://michaelnielsen.org/}{Michael Nielsen} 著}\\
  \vspace{1cm}
  {\Large
    \begin{tabular}{rl}
      \href{mailto:xhzhu.nju@gmail}{Xiaohu Zhu} & \multirow{2}{*}{译} \\
      \href{mailto:zhanggyb@gmail.com}{Freeman Zhang} & \\
    \end{tabular}
  }\\
  \vfill
  {\large \today}\\
  \vspace{1em}
  {\Huge\textcolor{red}{Early Preview}}
\end{center}
\end{titlepage}
