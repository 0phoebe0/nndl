% file: translators.tex

\chapter{关于本书翻译}
\label{ch:TranslationTeam}

\section*{开放源码项目}

这本书最初是我学
习 \href{http://neuralnetworksanddeeplearning.com/index.html}{Neural Networks
  and Deep Learning} 时做的中文笔记,因为原书中有很多数学公式,所以我
用 \LaTeX{}来编写和排版,并将所有 \LaTeX{} 源码放置
在 \href{https://github.com/zhanggyb/nndl}{GitHub}。其中部分内容取
自
\href{https://github.com/tigerneil/neural-networks-and-deep-learning-zh-cn}{Xiaohu
  Zhu 已经完成的翻译}来避免重复的工作。

第三、第四章的部分内容和\href{http://neuralnetworksanddeeplearning.com}{原文}略有
不同。原文中这两章的部分图形提供了交互式的形式,而在这本中文版中则全部换成了静态
图形。你可以在\href{http://neuralnetworksanddeeplearning.com}{原文}网页上试试调节
对应图像的可操作的参数加深理解,但仅阅读中文版本也不会有任何障碍。

如果你对此中译本有任何建议和意见,欢迎
以 \href{https://github.com/zhanggyb/nndl/issues}{issue} 的方式提交
到 \href{https://github.com/zhanggyb/nndl}{GitHub} 项目主页。

\begin{flushright}
  ——~Freeman Zhang
\end{flushright}

\section*{翻译团队}
\label{sec:TranslationTeam}

本书由以下人员翻译及整理:

\begin{itemize}
\item \textbf{\href{mailto:xhzhu.nju@gmail}{Xiaohu Zhu}}:翻译第二、三、五、六章
  内容。
\item \textbf{\href{mailto:zhanggyb@gmail.com}{Freeman Zhang}}:(正在)翻译其余章节、校
  对、整理及排版。
\item \textbf{\href{https://github.com/zhanggyb/nndl/graphs/contributors}{其他贡献者}}
\end{itemize}
