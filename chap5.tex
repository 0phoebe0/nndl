% file: chap5.tex

\chapter{深度神经网络为何很难训练}
\label{ch:WhyHardToTrain}

假设你是一名工程师,接到一项从头开始设计计算机的任务。某天,你在工作室工作,设计逻辑电路,构建 $AND$ 门,$OR$ 门等等时,老板带着坏消息进来:客户刚刚添加了一个奇特的设计需求:整个计算机的线路的深度必须只有两层:
\begin{center}
  \includegraphics{shallow_circuit}
\end{center}

你惊呆了,跟老板说道:“这货疯掉了吧!” 
 
老板说:“他们确实疯了,但是客户的需求比天大,我们要满足它。” 
 
实际上,在某种程度上看,他们的客户并没有太疯狂。假设你可以使用贵重特殊的逻辑门可以 $AND$ 起来你想要的那么多的输入。同样也能使用多值输入的 $NAND$ 门——可以 $AND$ 多个输入然后求否定的门。有了这类特殊的门,构建出来的两层的深度的网络便可以计算任何函数。 
但是仅仅因为某件事是理论上可能的,就代表这是一个好的想法。在实践中,在解决线路设计问题(或者大多数的其他算法问题)时,我们通常考虑如何解决子问题,然后逐步地集成这些子问题的解。换句话说,我们通过多层的抽象来获得最终的解答。 

例如,我们来设计一个逻辑线路来做两个数的乘法。我们希望在已经有了计算两个数加法的子线路基础上创建这个逻辑线路。计算两个数和的子线路也是构建在用语两个比特相加的子子线路上的。最终的线路就长成这个样子: 

\section{消失的梯度问题}
\label{sec:the_vanishing_gradient_problem}

\section{什么导致了消失的梯度问题?也就是在深度神经网络中的所谓的梯度不稳定性}

\section{在更加复杂网络中的不稳定梯度}

\section{其它深度学习的障碍}