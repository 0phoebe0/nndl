%% glossaries

%% Chapter 1

\newglossaryentry{perceptron}{
  name={感知机},
  description={\emph{Perceptron}}
}

\newglossaryentry{sigmoid-neuron}{
  name={\ S 型神经元},
  description={\emph{Sigmoid Neuron}}
}

\newglossaryentry{sigmoid-func}{
  name={\ S 型函数},
  description={\emph{Sigmoid Function}}
}

\newglossaryentry{sgd}{
  name={随机梯度下降},
  description={\emph{Stochastic Gradient Descent}}
}

\newglossaryentry{weight}{
  name={权重},
  description={\emph{Weight}}
}

\newglossaryentry{bias}{
  name={偏置},
  description={\emph{Bias}}
}

\newglossaryentry{threshold}{
  name={阈值},
  description={\emph{Threshold}}
}

\newglossaryentry{epoch}{
  name={迭代期},
  description={\emph{Epoch}}
}

\newglossaryentry{mini-batch}{
  name={小批量数据},
  description={\emph{Mini-batch}}
}

\newglossaryentry{hidden-layer}{
  name={隐藏层},
  description={\emph{Hidden Layer}}
}

\newglossaryentry{mlp}{
  name={多层感知机},
  description={\emph{Multilayer Perceptron}}
}

\newglossaryentry{cost-func}{
  name={代价函数},
  description={Cost Function}
}

\newglossaryentry{learning-rate}{
  name={学习速率},
  description={\emph{Learning Rate}}
}

\newglossaryentry{bp}{
  name={反向传播},
  description={\emph{Backpropagation}}
}

\newglossaryentry{svm}{
  name={支持向量机},
  description={\emph{Support Vector Machine}}
}

\newglossaryentry{error}{
  name={误差},
  description={\emph{Error}}
}

\newglossaryentry{reln}{
  name={修正线性神经元},
  description={\emph{Rectified Linear Neuron}}
}

\newglossaryentry{relu}{
  name={修正线性单元},
  description={\emph{Rectified Linear Unit}}
}

\newglossaryentry{softmax}{
  name={柔性最大值},
  description={Softmax}
}

\newglossaryentry{softmax-func}{
  name={柔性最大值函数},
  description={Softmax Function}
}

%% ------

\newglossaryentry{logistic-regression}{
  name={逻辑回归},
  description={\emph{Logistic Regression}}
}

\newglossaryentry{naive-bayes}{
  name={朴素贝叶斯},
  description={\emph{naive Bayes}}
}

\newglossaryentry{representations}{
  name={表征},
  description={\emph{Representations},表征是信息的呈现方式}
}

\newglossaryentry{rep-learning}{
  name={表征学习},
  description={\emph{Representation Learning}}
}

\newglossaryentry{autoencoder}{
  name={自编码器},
  description={\emph{Autoencoder (s)}}
}

\newglossaryentry{encoder}{
  name={编码器},
  description={\emph{encoder}}
}

\newglossaryentry{decoder}{
  name={解码器},
  description={\emph{decoder}}
}

\newglossaryentry{fov}{
  name={变化因素},
  description={\emph{factors of variation}}
}

%% Chapter 2

\newglossaryentry{scalar}{
  name={标量},
  description={scalar}
}

\newglossaryentry{scalars}{
  name={标量},
  description={Scalars}
}

\newglossaryentry{vec}{
  name={向量},
  description={vector}
}

\newglossaryentry{vecs}{
  name={向量},
  description={Vectors}
}

\newglossaryentry{matrix}{
  name={矩阵},
  description={matrix}
}

\newglossaryentry{matrices}{
  name={矩阵},
  description={Matrices}
}

\newglossaryentry{tensor}{
  name={张量},
  description={tensor}
}

\newglossaryentry{tensors}{
  name={张量},
  description={Tensors}
}

\newglossaryentry{transpose}{
  name={转置},
  description={transpose}
}

\newglossaryentry{main-diag}{
  name={主对角线},
  description={main diagonal}
}

\newglossaryentry{broadcasting}{
  name={广播},
  description={broadcasting}
}

\newglossaryentry{matrix-product}{
  name={矩阵积},
  description={matrix product}
}

\newglossaryentry{element-product}{
  name={按元素乘积},
  description={element-wise product}
}

\newglossaryentry{hadamard-product}{
  name={阿达玛乘积},
  description={Hadamard product}
}

\newglossaryentry{dot-product}{
  name={点乘},
  description={dot product}
}

\newglossaryentry{matrix-inversion}{
  name={矩阵求逆},
  description={matrix inversion}
}

\newglossaryentry{identity-matrix}{
  name={单位矩阵},
  description={identity matrix}
}

\newglossaryentry{linear-comb}{
  name={线性组合},
  description={linear combination}
}

\newglossaryentry{span}{
  name={生成空间},
  description={span}
}

\newglossaryentry{column-space}{
  name={列空间},
  description={column space}
}

\newglossaryentry{range}{
  name={范围},
  description={range}
}

\newglossaryentry{linear-dep}{
  name={线性相关},
  description={linear dependence}
}

\newglossaryentry{linearly-dep}{
  name={线性无关},
  description={linearly dependent}
}

\newglossaryentry{linearly-indep}{
  name={线性无关},
  description={linearly independent}
}

\newglossaryentry{linear-indep}{
  name={线性无关},
  description={linear independent}
}

\newglossaryentry{square}{
  name={方的},
  description={square}
}

\newglossaryentry{singular}{
  name={奇异矩阵},
  description={singular}
}

\newglossaryentry{norm}{
  name={范数},
  description={norm}
}

\newglossaryentry{tri-inequal}{
  name={三角不等式},
  description={triangle inequality}
}

\newglossaryentry{eu-norm}{
  name={欧几里德范数},
  description={Euclidean norm}
}

\newglossaryentry{max-norm}{
  name={最大值范数},
  description={max norm}
}

\newglossaryentry{fr-norm}{
  name={弗罗贝尼乌斯范数},
  description={Frobenius norm}
}

\newglossaryentry{diag}{
  name={对角线},
  description={Diagonal}
}

\newglossaryentry{symmetric}{
  name={对称},
  description={symmetric}
}

\newglossaryentry{unit-vec}{
  name={单位向量},
  description={unit vector}
}

\newglossaryentry{unit-norm}{
  name={单位范数},
  description={unit norm}
}

\newglossaryentry{ortho}{
  name={正交的},
  description={orthogonal}
}

\newglossaryentry{orthonormal}{
  name={标准正交的},
  description={orthonormal}
}

\newglossaryentry{orthonormal-matrix}{
  name={正交矩阵},
  description={orthonormal matrix}
}

\newglossaryentry{eigen-decompos}{
  name={特征分解},
  description={eigendecomposition}
}

\newglossaryentry{eigen-vec}{
  name={特征向量},
  description={eigenvector}
}

\newglossaryentry{eigen-vecs}{
  name={特征向量},
  description={eigenvectors}
}

\newglossaryentry{eigen-val}{
  name={特征值},
  description={eigenvalue}
}

\newglossaryentry{eigen-vals}{
  name={特征值},
  description={eigenvalues}
}

\newglossaryentry{left-eigen-vec}{
  name={左特征向量},
  description={left eigenvector}
}

\newglossaryentry{positive-definite}{
  name={正定矩阵},
  description={positive definite}
}

\newglossaryentry{positive-semidefinite}{
  name={半正定矩阵},
  description={positive semidefinite}
}

\newglossaryentry{negative-definite}{
  name={负定矩阵},
  description={negative definite}
}

\newglossaryentry{negative-semidefinite}{
  name={半负定矩阵},
  description={negative semidefinite}
}

\newglossaryentry{svd}{
  name={奇异值分解},
  description={singular value decomposition}
}

\newglossaryentry{singular-vecs}{
  name={奇异向量},
  description={singular vectors}
}

\newglossaryentry{singular-vals}{
  name={奇异值},
  description={singular values}
}

\newglossaryentry{singular-val}{
  name={奇异值},
  description={singular value}
}

\newglossaryentry{left-singular-vecs}{
  name={左奇异向量},
  description={left-singular vectors}
}

\newglossaryentry{right-singular-vecs}{
  name={右奇异向量},
  description={right-singular vectors}
}

\newglossaryentry{moore-penrose-pseudoinverse}{
  name={摩尔--彭若斯广义逆},
  description={Moore-Penrose pseudoinverse}
}

\newglossaryentry{pca}{
  name={主成分分析},
  description={principal components analysis}
}

%% Chapter 4

\newglossaryentry{overflow}{
  name={溢出},
  description={overflow}
}

\newglossaryentry{underflow}{
  name={下溢},
  description={underflow}
}

\newglossaryentry{cond-num}{
  name={条件数},
  description={condition number}
}

\newglossaryentry{obj-func}{
  name={目标函数},
  description={objective function}
}

\newglossaryentry{criterion}{
  name={准则},
  description={criterion, criterion funciton 准则函数}
}

\newglossaryentry{loss-func}{
  name={损失函数},
  description={loss function}
}

\newglossaryentry{err-func}{
  name={误差函数},
  description={error function}
}

\newglossaryentry{gradient-descent}{
  name={梯度下降},
  description={gradient descent}
}

\newglossaryentry{critical-points}{
  name={临界点},
  description={critical points}
}

\newglossaryentry{stationary-points}{
  name={驻点},
  description={stationary points}
}

\newglossaryentry{local-min}{
  name={局部最小值},
  description={local minimum}
}

\newglossaryentry{local-max}{
  name={局部最大值},
  description={local maximum}
}

\newglossaryentry{saddle-points}{
  name={鞍点},
  description={saddle points}
}

\newglossaryentry{global-min}{
  name={全局最小值},
  description={global minimum}
}

\newglossaryentry{partial-derivatives}{
  name={偏导数},
  description={partial derivatives}
}

\newglossaryentry{gradient}{
  name={梯度},
  description={gradient}
}

\newglossaryentry{directional-derivative}{
  name={方向导数},
  description={directional derivative}
}

\newglossaryentry{steepest-descent}{
  name={最陡下降法},
  description={method of steepest descent}
}

\newglossaryentry{line-search}{
  name={线搜索},
  description={line search}
}

\newglossaryentry{hill-climbing}{
  name={爬山算法},
  description={hill climbing}
}

\newglossaryentry{jacobian-matrix}{
  name={雅可比矩阵},
  description={Jacobian matrix}
}

\newglossaryentry{second-derivative}{
  name={二阶导数},
  description={second derivative}
}

\newglossaryentry{curvature}{
  name={曲率},
  description={curvature}
}

\newglossaryentry{hessian-matrix}{
  name={海森矩阵},
  description={Hessian matrix}
}

\newglossaryentry{second-derivative-test}{
  name={二阶导数检测},
  description={second derivative test}
}

%% Chapter 12

\newglossaryentry{warps}{
  name={线程束},
  description={\emph{Warps},同时运行的一组线程的称呼}
}

\newglossaryentry{overfitting}{
  name={过度拟合},
  description={\emph{overfitting},过度拟合,过拟合,过适}
}

\newglossaryentry{generalization_error}{
  name={泛化误差},
  description={\emph{Generalization error},泛化误差}
}

\newglossaryentry{dropout}{
  name={弃权},
  description={\emph{Dropout}, 弃权}
}

\newglossaryentry{bdt}{
  name={提高决策树},
  description={\emph{Boosted decision trees}}
}

\newglossaryentry{gcn}{
  name={全局对比度归一化},
  description={\emph{Global contrast normalization}, 全局对比度归一化}
}

\newglossaryentry{mode}{
  name={众数},
  description={\emph{mode}, 众数}
}
