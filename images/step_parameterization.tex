%!TEX TS-program = xelatex
%!TEX encoding = UTF-8 Unicode

\documentclass[11pt,tikz,border=1]{standalone}
\usetikzlibrary{calc,positioning}
\usepackage{pgfplots}

\usepackage{cjkfonts}

\begin{document}

  \begin{tikzpicture}[
    neuron/.style={circle,draw,inner sep=0pt,minimum size=10mm}
    ]

    \begin{scope}
      
      \node (l0) [neuron] {$x$};
      \node (m0) [neuron,right=of l0,yshift=-1.5cm] {};
      \node (m1) [neuron,right=of l0,yshift=1.5cm] {};
      \node (r0) [neuron,right=of m0,yshift=1.5cm] {};

      \draw[->] (r0) -- ++(1,0);

      \draw[->] (l0) to (m0);
      \draw[->] (l0) to (m1);
      \draw[->] (m0) to (r0);
      \draw[->] (m1) to (r0);

      \node(b) [blue,above] at (m1.north) {$s = 0.40$};
    \end{scope}

    \begin{scope}[xshift=6cm]

      \begin{axis}[
          width=5.6cm,
          height=5.6cm,
          xlabel={\normalsize $x$},
          axis lines=left,
          tick label style={font=\tiny},
          label style={font=\tiny},
          title style={font=\scriptsize},
          xtick distance=1,
          ytick distance=1,
          xmax=1.1,
          ymax=1.1,
          title={顶部隐藏神经元的输出}
        ]
        \addplot[
          orange,
          thick,
          domain=0:1,
          samples=101
        ] {
          1/(1 + exp(-(1000 * x + (-0.40 * 1000))))        
        };
      \end{axis}
      
    \end{scope}
    
  \end{tikzpicture}

\end{document}
