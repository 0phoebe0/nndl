% file: plots.tex
% used for plotting learning curve in chapter 3

% learn cost curve of a single sigmoid neutron with quadratic cost function
\newcommand{\quadraticCostLearning}[4]{
  \begin{tikzpicture}[
    inner sep=0pt,
    minimum size=10mm,
    background rectangle/.style={
      draw=gray!25,
      fill=gray!10,
      rounded corners
    },
    show background rectangle]

    \coordinate (origin) at (0,0);
    \coordinate(x) at (3.5,0);
    \coordinate(y) at (0,2.5);

    \node(i) [above=4 of origin,xshift=-5mm] {\footnotesize Input: $1.0$};
    \node(n) [right=2 of i,circle,draw] {};
    \node(epoch) [right=of x,xshift=-1cm] {\footnotesize Epoch};
    \node(cost) [left=of y,xshift=1cm] {\footnotesize Cost};

    \draw[->] (origin) to (x);
    \draw[->] (origin) to (y);

    %\draw[blue,thick,domain=0:2] plot (\x, {(\x*\x / 2)});

    \tikzmath{                  % Do not contain blank line
      function sigmoid(\z) {
        return 1/(1 + exp(- \z));
      };
      function sigmoid_neutron(\w,\b) {
        return sigmoid(\w + \b);
      };
      function quad_cost(\a) {         % the quadratic cost function
        return (\a * \a)/2;
      };
      function quad_cost_derivative(\a) {
        return \a * \a * (1-\a);
      };
      \w = #1;                % start weight
      \b = #2;                % start bias
      \e = #3;                % eta, learning rate
      \y = 0;
      \a = sigmoid_neutron(\w,\b);
      \dt = quad_cost_derivative(\a);
      \xo = 0;
      \yo = quad_cost(\a);
      integer \x;
      if #4 > 0 then {
        for \x in {1,...,#4}{  % epoches
          \a = sigmoid_neutron(\w,\b);
          \dt = quad_cost_derivative(\a);
          \w = \w - \e * \dt;
          \b = \b - \e * \dt;
          \y = quad_cost(\a);
          {\draw[blue,thick] (\xo/100,\yo*5) -- (\x/100,\y*5);}; % scale y with 5 times
          \xo = \xo + 1;
          \yo = \y;
        };
        {\draw (\x/100,0) -- node[below] {\footnotesize $\x$} (\x/100,-0.1);};
      };
      {
        \draw[->] (i) to node (w) [below] {
          \scriptsize
          \pgfkeys{/pgf/number format/.cd,showpos,fixed,fixed zerofill,precision=2,use period}
          $w = \pgfmathprintnumber{\w}$
        } (n);
        \node(b) [below,xshift=5mm] at (n.south) {
          \scriptsize
          \pgfkeys{/pgf/number format/.cd,showpos,fixed,fixed zerofill,precision=2,use period}
          $b = \pgfmathprintnumber{\b}$
        };
        \node(o) [right=of n] {
          \footnotesize
          \pgfkeys{/pgf/number format/.cd,fixed,fixed zerofill,precision=2,use period}
          Output: $\pgfmathprintnumber{\a}$
        };
        \draw[->] (n) to (o);
      };
    }

  \end{tikzpicture}%
}

% learn cost curve of a single sigmoid neutron with cross entropy cost function
\newcommand{\crossEntropyCostLearning}[4]{
  \begin{tikzpicture}[
    inner sep=0pt,
    minimum size=10mm,
    background rectangle/.style={
      draw=gray!25,
      fill=gray!10,
      rounded corners
    },
    show background rectangle]

    \coordinate (origin) at (0,0);
    \coordinate(x) at (3.5,0);
    \coordinate(y) at (0,2.5);

    \node(i) [above=4 of origin,xshift=-5mm] {\footnotesize Input: $1.0$};
    \node(n) [right=2 of i,circle,draw] {};
    \node(epoch) [right=of x,xshift=-1cm] {\footnotesize Epoch};
    \node(cost) [left=of y,xshift=1cm] {\footnotesize Cost};

    \draw[->] (origin) to (x);
    \draw[->] (origin) to (y);

    %\draw[blue,thick,domain=0:2] plot (\x, {(\x*\x / 2)});

    \tikzmath{                  % Do not contain blank line
      function sigmoid(\z) {
        return 1/(1 + exp(- \z));
      };
      function sigmoid_neutron(\w,\b) {
        return sigmoid(\w + \b);
      };
      function cross_entropy_cost(\a) {         % the quadratic cost function
        return -ln(1 - \a);
      };
      function cross_entropy_cost_derivative(\a) {
        return 1/(1 - \a);
      };
      \w = #1;                % start weight
      \b = #2;                % start bias
      \e = #3;                % eta, learning rate
      \y = 0;
      \a = sigmoid_neutron(\w,\b);
      \dt = cross_entropy_cost_derivative(\a);
      \xo = 0;
      \yo = cross_entropy_cost(\a);
      integer \x;
      if #4 > 0 then {
        for \x in {1,...,#4}{  % epoches
          \a = sigmoid_neutron(\w,\b);
          \dt = cross_entropy_cost_derivative(\a);
          \w = \w - \e * \dt;
          \b = \b - \e * \dt;
          \y = cross_entropy_cost(\a);
          {\draw[blue,thick] (\xo/100,\yo/2) -- (\x/100,\y/2);}; % scale y to 1/2
          \xo = \xo + 1;
          \yo = \y;
        };
        {\draw (\x/100,0) -- node[below] {\footnotesize $\x$} (\x/100,-0.1);};
      };
      {
        \draw[->] (i) to node (w) [below] {
          \scriptsize
          \pgfkeys{/pgf/number format/.cd,showpos,fixed,fixed zerofill,precision=2,use period}
          $w = \pgfmathprintnumber{\w}$
        } (n);
        \node(b) [below,xshift=5mm] at (n.south) {
          \scriptsize
          \pgfkeys{/pgf/number format/.cd,showpos,fixed,fixed zerofill,precision=2,use period}
          $b = \pgfmathprintnumber{\b}$
        };
        \node(o) [right=of n] {
          \footnotesize
          \pgfkeys{/pgf/number format/.cd,fixed,fixed zerofill,precision=2,use period}
          Output: $\pgfmathprintnumber{\a}$
        };
        \draw[->] (n) to (o);
      };
    }

  \end{tikzpicture}%
}
